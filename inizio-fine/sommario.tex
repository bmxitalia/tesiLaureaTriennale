% !TEX encoding = UTF-8
% !TEX TS-program = pdflatex
% !TEX root = ../tesi.tex

%**************************************************************
% Sommario
%**************************************************************
\cleardoublepage
\phantomsection
\pdfbookmark{Sommario}{Sommario}
\begingroup
\let\clearpage\relax
\let\cleardoublepage\relax
\let\cleardoublepage\relax

\chapter*{Sommario}

Il presente documento descrive il lavoro svolto durante il periodo di stage, della durata di 320 ore, dal laureando Tommaso Carraro presso l'azienda \visione{} S.r.l. situata a Pernumia (PD).
Lo scopo principale del progetto era la realizzazione di un'applicazione mobile che permettese ai clienti di una qualsiasi azienda di acquistare, tramite il proprio smartphone, dei prodotti venduti dalla stessa. Per raggiungere questo fine sono stati assegnati vari compiti.
Lo studente ha dovuto scegliere in autonomia l'ambiente di sviluppo ritenuto più opportuno. Poiché era richiesto che l'applicazione funzionasse sia in ambiente Android che in ambiente iOS, si è dovuto scegliere un framework cross-platfrom e, in particolare, il framework PhoneGap.
In seguito alla scelta del framework vi è stato un periodo di formazione, di circa 40 ore, sui software gestionali utilizzati in azienda e sul linguaggio JavaScript, in modo da facilitare lo sviluppo del progetto.
Si è poi potuto proseguire con la progettazione delle varie componenti dell'applicazione, quali servizio web, database sottostanti, logica applicativa e interfaccia grafica. 
scrivere che il servizio e i database dovevano stare su un server cloud e che il database aziendale poteva essere locale o remoto al server
In particolare, è stato richiesto di progettare un database che permettesse la gestione dei dati di autenticazione di moviORDER e un database che contenesse i dati utili alla gestione degli ordini presso un'azienda. 
Spiegare i compiti del servizio, in particolare richieste e risposte.
Spiegare come doveva essere l'interfaccia grafica di moviORDER e come doveva essere la logica applicativa.
In secondo luogo era richiesta l'implementazione di un ... 
Tale framework permette di registrare gli eventi di un controllore programmabile, quali segnali applicati 
Terzo ed ultimo obbiettivo era l'integrazione ...

%\vfill
%
%\selectlanguage{english}
%\pdfbookmark{Abstract}{Abstract}
%\chapter*{Abstract}
%
%\selectlanguage{italian}

\endgroup			

\vfill

