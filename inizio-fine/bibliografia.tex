% !TEX encoding = UTF-8
% !TEX TS-program = pdflatex
% !TEX root = ../tesi.tex

%**************************************************************
% Bibliografia
%**************************************************************

\cleardoublepage
\chapter{Bibliografia}

\begin{enumerate}[label={[\arabic*]}]
	\item \textit{Agile, definizione} - \url{https://en.wikipedia.org/wiki/Agile_software_development}
	\item \textit{AJAX, definizione} - \url{https://en.wikipedia.org/wiki/Ajax_(programming)}
	\item \textit{API, definizione} - \url{https://en.wikipedia.org/wiki/Application_programming_interface}
	\item \textit{API testing, definizione} - \url{https://en.wikipedia.org/wiki/API_testing}
	\item \textit{Applicazione web, definizione} - \url{https://en.wikipedia.org/wiki/Web_application}
	\item \textit{Architettura, definizione} - \url{https://en.wikipedia.org/wiki/Software_architecture}
	\item \textit{Architettura a strati, definizione} - \url{https://en.wikipedia.org/wiki/Multitier_architecture}
	\item \textit{Attore, definizione} - \url{https://www.math.unipd.it/~tullio/IS-1/2017/Dispense/E02.pdf}
	\item \textit{Azure, definizione} - \url{https://en.wikipedia.org/wiki/Microsoft_Azure}
	\item \textit{Back end, definizione} - \url{https://en.wikipedia.org/wiki/Front_and_back_ends}
	\item \textit{Baseline, definizione} - \url{https://www.math.unipd.it/~tullio/IS-1/2017/Dispense/L07.pdf}
	\item \textit{Big-bang-integration, definizione} - \url{https://en.wikipedia.org/wiki/Integration_testing} 
	\item \textit{C\#, definizione} - \url{https://en.wikipedia.org/wiki/C_Sharp_(programming_language)}
	\item \textit{Caso d’uso, definizione} - \url{https://en.wikipedia.org/wiki/Use_case}
	\item \textit{Ciclo di vita, definizione} - \url{https://en.wikipedia.org/wiki/Software_development_process}
	\item \textit{Cloud computing, definizione} - \url{https://en.wikipedia.org/wiki/Cloud_computing}
	\item \textit{Code-n-fix, definizione} - \url{https://it.wikipedia.org/wiki/Code_and_fix}
	\item \textit{Codice nativo, definizione} - \url{https://searchmicroservices.techtarget.com/definition/native-code}
	\item \textit{Codice oggetto, definizione} - \url{https://en.wikipedia.org/wiki/Object_code}
	\item \textit{Commit, definizione} - \url{https://en.wikipedia.org/wiki/Commit_(version_control)}
	\item \textit{Community, definizione} - \url{https://en.wikipedia.org/wiki/Community}
	\item \textit{Compile-time, definizione} - \url{https://en.wikipedia.org/wiki/Compile_time}
	\item \textit{Container, definizione} - \url{https://www.docker.com/resources/what-container}
	\item \textit{Correttezza per costruzione, definizione} - \url{https://www.us-cert.gov/bsi/articles/knowledge/sdlc-process/correctness-by-construction}
	\item \textit{Cross-compiled, definizione - R. Raj, S. B. Tolety. A study on approaches to build cross-platform mobile applications and criteria to select appropriate approach}
	\item \textit{CSS, definizione} - \url{https://en.wikipedia.org/wiki/Cascading_Style_Sheets}
	\item \textit{DBMS, definizione} - \url{https://it.wikipedia.org/wiki/Database_management_system}
	\item \textit{Deploy, definizione} - \url{https://en.wikipedia.org/wiki/Software_deployment}
	\item \textit{Design pattern, definizione} - \url{https://en.wikipedia.org/wiki/Software_design_pattern}
	\item \textit{Desktop publishing, definizione} - \url{https://en.wikipedia.org/wiki/Desktop_publishing}
	\item \textit{Diagramma di Gantt, definizione} - \url{https://it.wikipedia.org/wiki/Diagramma_di_Gantt}
	\item \textit{Dialog, definizione} - \url{https://en.wikipedia.org/wiki/Dialog_box}
	\item \textit{E-business, definizione} - \url{https://en.wikipedia.org/wiki/Electronic_business}
	\item \textit{End-point, definizione} - \url{https://en.wikipedia.org/wiki/Communication_endpoint}
	\item \textit{ER, definizione} - \url{https://www.smartdraw.com/entity-relationship-diagram/}
	\item \textit{ERP, definizione} - \url{https://it.wikipedia.org/wiki/Enterprise_resource_planning}
	\item \textit{Framework, definizione} - \url{https://www.math.unipd.it/~tullio/IS-1/2017/Dispense/L10.pdf}
	\item \textit{Framework cross-platform, definizione} - \url{https://www.altexsoft.com/blog/engineering/xamarin-vs-react-native-vs-ionic-vs-nativescript-cross-platform-mobile-frameworks-comparison/}
	\item \textit{Front end, definizione} - \url{https://en.wikipedia.org/wiki/Front_and_back_ends}
	\item \textit{Funzione anonima, definizione} - \url{https://en.wikipedia.org/wiki/Anonymous_function}
	\item \textit{HTML, definizione} - \url{https://en.wikipedia.org/wiki/HTML}
	\item \textit{HTTP, definizione} - \url{https://it.wikipedia.org/wiki/Hypertext_Transfer_Protocol}
	\item \textit{ICT, definizione} - \url{https://en.wikipedia.org/wiki/Information_and_communications_technology}
	\item \textit{IDE, definizione} - \url{https://en.wikipedia.org/wiki/Integrated_development_environment}
	\item \textit{Interprete, definizione} - \url{https://en.wikipedia.org/wiki/Interpreter_(computing)}
	\item \textit{JavaServer Pages, definizione} - \url{https://en.wikipedia.org/wiki/JavaServer_Pages}
	\item \textit{JDBC, definizione} - \url{https://en.wikipedia.org/wiki/Java_Database_Connectivity}
	\item \textit{JQuery, definizione} - \url{https://en.wikipedia.org/wiki/JQuery}
	\item \textit{JSON, definizione} - \url{https://en.wikipedia.org/wiki/JSON}
	\item \textit{Libreria, definizione} - \url{https://en.wikipedia.org/wiki/Library_(computing)}
	\item \textit{Media queries, definizione} - \url{https://en.wikipedia.org/wiki/Media_queries}
	\item \textit{Milestone, definizione} - \url{https://www.math.unipd.it/~tullio/IS-1/2017/Dispense/L07.pdf}
	\item \textit{Mobile first, definizione} - \url{https://en.ryte.com/wiki/Mobile_First}
	\item \textit{Modal, definizione} - \url{https://en.wikipedia.org/wiki/Modal_window}
	\item \textit{MoviDOC, definizione - Microanalisi ricevuta dal tutor aziendale}
	\item \textit{Multipiattaforma, definizione} - \url{https://it.wikipedia.org/wiki/Multipiattaforma}
	\item \textit{MVC, definizione} - \url{https://https://en.wikipedia.org/wiki/Model-view-controller}
	\item \textit{MVP, definizione} - \url{https://en.wikipedia.org/wiki/Model-view-presenter}
	\item \textit{Objective-C++, definizione}  - \url{https://en.wikipedia.org/wiki/Objective-C}
	\item \textit{Parsing, definizione} - \url{https://it.wikipedia.org/wiki/Parsing}
	\item \textit{Periodo di slack, definizione} - \url{https://searchsap.techtarget.com/answer/What-is-slack-time}
	\item \textit{Phablet, definizione} - \url{https://en.wikipedia.org/wiki/Phablet}
	\item \textit{PHP, definizione} - \url{https://en.wikipedia.org/wiki/PHP}
	\item \textit{Piattaforma, definizione} - \url{https://it.wikipedia.org/wiki/Piattaforma_(informatica)}
	\item \textit{Refactoring, definizione} - \url{https://it.wikipedia.org/wiki/Refactoring}
	\item \textit{Responsive, definizione} - \url{https://it.wikipedia.org/wiki/Design_responsivo}
	\item \textit{Robustezza, definizione} - \url{https://it.wikipedia.org/wiki/Robustezza_(informatica)}
	\item \textit{Run-time, definizione} - \url{https://en.wikipedia.org/wiki/Run_time_(program_lifecycle_phase)}
	\item \textit{Server cloud, definizione} - \url{https://it.wikipedia.org/wiki/Cloud_Server}
	\item \textit{Server web, definizione} - \url{https://it.wikipedia.org/wiki/Server_web} 
	\item \textit{Servizio web, definizione} - \url{https://it.wikipedia.org/wiki/Web_service} 
	\item \textit{Servlet, definizione} - \url{https://en.wikipedia.org/wiki/Java_servlet}
	\item \textit{SMTP, definizione} - \url{https://en.wikipedia.org/wiki/Simple_Mail_Transfer_Protocol} 
	\item \textit{Software gestionale, definizione} - \url{https://it.wikipedia.org/wiki/Software_gestionale}
	\item \textit{Tecnologie web, definizione} - \url{https://en.wikipedia.org/wiki/Web_development}
	\item \textit{Ticketing, definizione} - \url{https://www.math.unipd.it/~tullio/IS-1/2017/Dispense/L07.pdf}
	\item \textit{Tracciabilità, definizione} - \url{https://it.wikipedia.org/wiki/Tracciabilità_(informatica)} 
	\item \textit{Transact-SQL, definizione} - \url{https://it.wikipedia.org/wiki/Transact-SQL} 
	\item \textit{UML, definizione} - \url{https://en.wikipedia.org/wiki/Unified_Modeling_Language} 
	\item \textit{Usabilità, definizione} - \url{https://en.wikipedia.org/wiki/Usability}
\end{enumerate}
