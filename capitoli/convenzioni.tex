\chapter{Convenzioni} \label{convenzioni}

In questo capitolo vengono presentate le convenzioni utilizzate per la classificazione di casi d'uso e requisiti.

\section{Casi d'uso}

Ogni caso d'uso è classificato tramite il seguente formalismo:
\begin{center}
	\textbf{UC$\{$codice\_identificativo$\}$}
\end{center}

dove:

\begin{itemize}
	\item \textbf{codice\_identificativo}: è un codice composto da una serie di numeri separati tramite punto, che identificano il caso d'uso in maniera univoca e lo esprimono gerarchicamente.
\end{itemize}

\section{Requisiti}

Ogni requisito è classificato tramite il seguente formalismo:
\begin{center}
	\textbf{R$\{$X$\}$$\{$Y$\}$$\{$codice\_identificativo$\}$}
\end{center}

dove:

\begin{itemize}
	\item \textbf{X} specifica la tipologia di requisito:
        \begin{itemize}
            \item \textit{F}: requisito funzionale;
            \item \textit{Q}: requisito qualitativo;
            \item \textit{V}: requisito di vincolo.
        \end{itemize}
    \item \textbf{Y} indica uno dei seguenti gradi di necessità:
        \begin{itemize}
            \item \textit{O}: requisito obbligatorio;
            \item \textit{D}: requisito desiderabile;
            \item \textit{F}: requisito facoltativo.
        \end{itemize}
    \item \textbf{codice\_identificativo}: è un codice composto da una serie di numeri separati tramite
    punto, che identificano il requisito in maniera univoca e lo esprimono gerarchicamente.
\end{itemize}    
