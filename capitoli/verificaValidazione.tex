% !TEX encoding = UTF-8
% !TEX TS-program = pdflatex
% !TEX root = ../tesi.tex

%**************************************************************
\chapter{Verifica e validazione}
\label{cap:verifica-validazione}
%**************************************************************

In questo capitolo vengono presentate le tecniche di verifica e validazione utilizzate durante il progetto.

\section{Verifica}

La verifica è il processo che si occupa di fornire prove oggettive che quanto prodotto in una fase di ciclo di vita di sviluppo del software soddisfi tutti i requisiti specificati per tale fase. Essa deve essere eseguita durante lo sviluppo, al fine di garantire consistenza, correttezza e completezza del prodotto software e della documentazione ad esso associata. È importante verificare ogni qualvolta che viene raggiunto un risultato intermedio tangibile, al fine di accertare che le attività di processo eseguite per raggiungere tale risultato non abbiano introdotto errori.

\subsection{Analisi statica}

L'analisi statica è una forma di verifica che studia le caratteristiche del codice sorgente e della documentazione ad esso associata senza richiedere l'esecuzione del prodotto in alcuna sua parte. Essa permette di verificare la conformità rispetto a norme progettuali, l'assenza di difetti e la presenza di proprietà positive nel codice. L'analisi statica risulta essenziale finché il sistema non raggiunge la sua completa disponibilità e deve essere applicata ad ogni prodotto di processo, al fine di verificare che quanto prodotto raggiunga la qualità attesa. Poiché il costo di rilevazione e di correzione degli errori cresce con l'avanzare del progetto, bisogna prestare attenzione nella scelta dell'approccio di verifica corretto. Nel caso del progetto si è cercato di utilizzare un approccio costruttivo, il quale consiste nel far accompagnare lo sviluppo con cicli di verifica, ottenendo la correttezza per costruzione.

\subsubsection{Analisi statica del codice}

\myparagraph{Code reading}

Per la logica applicativa dell'applicazione, essendo scritta in \textit{JavaScript}, e quindi non compilata, si è utilizzata la tecnica del \textit{code reading}. Essa consiste nell'esecuzione di un'attenta lettura individuale del codice per l'individuazione di errori e/o discrepanze con il progetto. Il lettore effettua mentalmente una pseudo-esecuzione del codice che lo conduce a verificarne la correttezza rispetto alle specifiche e agli standard adottati. Questa tecnica è risultata utile nell'analisi della sintassi \textit{JavaScript} poco conosciuta, ad esempio la sintassi dei \textit{plugin PhoneGap} o la sintassi per l'utilizzo di \textit{AJAX}.

\myparagraph{Analisi statica in compilazione}

Per il servizio web, essendo scritto il linguaggio \textit{Java}, si è potuta utilizzare l'analisi statica in compilazione. I compilatori, per poter generare il \glossaryItem{codice oggetto}, effettuano un'analisi statica del codice per verificare che un programma soddisfi particolari caratteristiche di correttezza statica. Tipici errori scovati a \textit{compile-time} sono:
\begin{itemize}
	\item presenza di nomi di identificatori non dichiarati;
	\item incoerenza tra tipi di variabili utilizzate in un'istruzione;
	\item incoerenza tra parametri formali e parametri attuali nelle chiamate a metodi e/o funzioni: l'incoerenza potrebbe essere sul tipo e/o il numero di parametri.
\end{itemize}

\subsubsection{Analisi statica della documentazione}

Per la verifica della documentazione, essendo lo stagista da solo, si è dovuta utilizzare la tecnica \textit{walkthrough}. Essa mira a rilevare la presenza di difetti eseguendo una lettura critica a largo spettro del prodotto in esame, questo perché vi è il sospetto che i difetti possano essere ovunque. Per l'esecuzione di un buon \textit{walkthrough} è necessario che esso venga pianificato e che le attività svolte vengano documentate. Si è scelta questa tecnica perché nonostante abbia tempi di verifica molto lunghi, produce spesso documenti corretti.

\subsection{Analisi dinamica}

L'analisi dinamica è una forma di verifica che consiste nell'effettuare test sul prodotto in esecuzione. Un test è una verifica dinamica del comportamento del programma, effettuata su un insieme finito di casi, selezionati dal dominio di tutte le esecuzioni possibili. Poiché il dominio di tutte le esecuzioni possibili è potenzialmente infinito, i test non garantiscono l'esaustività, quindi l'analisi dinamica può rilevare la presenza di difetti ma non è in grado di dimostrarne l'assenza.

\subsubsection{Test della logica applicativa}

Per il test della logica applicativa si è utilizzata la tecnica del \textit{debugging}. Essa consiste nella ricerca e nella correzione di difetti che sono causa di malfunzionamenti rilevati a \textit{run-time}. Per il \textit{debugging} del codice \textit{JavaScript} si è dovuta utilizzare la \textit{forza bruta}, ossia disseminare il codice di istruzioni di stampa per estrapolare quante più informazioni possibili sulla semantica del codice eseguito. Poiché \textit{JavaScript} risulta essere un linguaggio difficilmente \textit{debuggabile}, il \textit{debugging} della logica applicativa è stato inefficiente e dispendioso in termini di tempo.

\subsubsection{Test dell'API}

Per il test dell'\textit{API} di \textit{moviORDER} si è utilizzato il software \textit{Postman}, già precedentemente menzionato in sezione \ref{background}. Tale software permette di costruire una richiesta \textit{HTTP}, di inviarla ad un \textit{server} e di analizzare la risposta ricevuta. Nel caso del progetto sono state create richieste \textit{HTTP POST} con i parametri richiesti dall'\textit{API} e si sono successivamente analizzate le risposte in formato \textit{JSON} ricevute dal \textit{server}.
Quindi il software ha permesso di testare il comportamento del servizio web alla ricezione di richieste \textit{HTTP} corrette e scorrette.

\subsubsection{Test dell'applicazione}

Per il test dell'applicazione, e quindi dell'usabilità e del funzionamento dell'interfaccia grafica, si è utilizzato l'emulatore dell'\textit{IDE} \textit{Android Studio}. Per il test della logica dell'applicazione si è considerato:
\begin{itemize}
	\item funzionamento dei bottoni: si è verificato che alla pressione di un bottone venisse riscontrato il comportamento atteso;
	\item funzionamento del \textit{plugin} per la scansione dei codici a barre: si sono testate tutte le tipologie di codici a barre richieste dal tutor aziendale.
\end{itemize}

Per il test del layout responsive dell'applicazione si sono utilizzati gli strumenti per gli sviluppatori del browser \textit{Google Chrome}. In particolare si è testato il comportamento del layout su tutte le fasce di dispositivi, ovvero su schermi piccoli, medi e grandi (\glossaryItem{phablet}).

\section{Validazione}

La validazione è il processo che si occupa di confermare, mediante esame e fornitura di prove oggettive, che le specifiche del software siano conformi alle esigenze dell'utente e all'uso previsto, e che i requisiti che il software implementa siano soddisfatti in maniera consistente.
La validazione, a differenza della verifica, viene eseguita al termine del progetto e, affinché il prodotto venga accettato, è necessario eseguire continue verifiche durante il ciclo di vita dello stesso.

\subsection{Validazione di moviORDER}

La validazione di \textit{moviORDER} è stata eseguita dal tutor aziendale tramite test dell'applicazione installata sul proprio \textit{smartphone}. Il tutor ha analizzato i requisiti dell'applicazione e ha verificato che fossero tutti soddisfatti.