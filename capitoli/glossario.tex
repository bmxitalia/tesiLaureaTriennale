\chapter{Glossario} \label{gloss}

\textbf{\textit{Agile}} Lo sviluppo \textit{agile} è un approccio allo sviluppo del \textit{software} in base al quale i requisiti e le soluzioni si evolvono attraverso lo sforzo collaborativo di \textit{team} auto-organizzanti e interfunzionali e dei loro clienti / utenti finali. Promuove la pianificazione adattiva, lo sviluppo evolutivo, la consegna anticipata e il miglioramento continuo, inoltre incoraggia una risposta rapida e flessibile al cambiamento.

\textbf{\textit{AJAX}} \textit{AJAX} (\textit{Asynchronous JavaScript And XML}) è un insieme di tecniche di sviluppo \textit{web} che utilizzano tecnologie \textit{web} \textit{client-side} per creare applicazioni \textit{web} asincrone. Con \textit{AJAX}, le applicazioni \textit{web} possono inviare e recuperare dati da un \textit{server} in modo asincrono, senza interferire con la visualizzazione e il comportamento della pagina esistente.

\textbf{\textit{API}} In informatica, un'\textit{API} (\textit{Application Programming Interface}) è un insieme di definizioni di \textit{subroutine}, protocolli di comunicazione e strumenti per la creazione di \textit{software}. Generalmente, si tratta di un insieme di metodi di comunicazione chiaramente definiti tra vari componenti. Una buona \textit{API} semplifica lo sviluppo di un programma fornendo tutti i blocchi costitutivi, che vengono poi assemblati dal programmatore. Un'\textit{API} può essere scritta per un sistema \textit{web-based}, un sistema operativo, un sistema di \textit{database}, \textit{hardware} del \textit{computer} o una libreria \textit{software}.

\textbf{\textit{API testing}}

\textbf{Applicazione \textit{web}} Un'applicazione \textit{web} (\textit{web application} in inglese) è un programma \textit{client-server} dove il \textit{client} (compresa l'interfaccia utente e la logica lato \textit{client}) viene eseguito su un \textit{browser web}.

\textbf{Architettura}

\textbf{Architettura a strati}

\textbf{Attore}

\textbf{\textit{Azure}}

\textbf{\textit{Back end}}

\textbf{\textit{Baseline}}

\textbf{\textit{Big-bang-integration}}

\textbf{\textit{C\#}}

\textbf{Caso d'uso}

\textbf{Ciclo di vita}

\textbf{\textit{Cloud computing}}

\textbf{\textit{Code-n-fix}}

\textbf{Codice nativo}

\textbf{Codice oggetto}

\textbf{\textit{Commit}}

\textbf{\textit{Community}}

\textbf{\textit{Compile-time}}

\textbf{\textit{Container}}

\textbf{Correttezza per costruzione}

\textbf{\textit{Cross-compiled}}

\textbf{\textit{CSS}}

\textbf{\textit{DBMS}}

\textbf{\textit{Deploy}}

\textbf{\textit{Design pattern}}

\textbf{\textit{Desktop publishing}}

\textbf{Diagramma di \textit{Gantt}}

\textbf{\textit{Dialog}}

\textbf{\textit{E-business}}

\textbf{\textit{End-point}}

\textbf{\textit{ER}}

\textbf{\textit{ERP}}

\textbf{\textit{Framework}}

\textbf{\textit{Framework cross-platform}}

\textbf{\textit{Front end}}

\textbf{Funzione anonima}

\textbf{\textit{HTML}}

\textbf{\textit{HTTP}}

\textbf{\textit{ICT}}

\textbf{\textit{IDE}}

\textbf{Interprete}

\textbf{\textit{JavaServer Pages}}

\textbf{\textit{JDBC}}

\textbf{\textit{JQuery}}

\textbf{\textit{JSON}}

\textbf{\textit{Layout} elastico}

\textbf{Libreria}

\textbf{\textit{Media queries}}

\textbf{\textit{Milestone}}

\textbf{\textit{Mobile first}}

\textbf{\textit{Modal}}

\textbf{\textit{Model}}

\textbf{\textit{MoviDOC}}

\textbf{Multipiattaforma}

\textbf{\textit{MVC}}

\textbf{\textit{MVP}}

\textbf{\textit{Objective-C++}}

\textbf{\textit{Parsing}}

\textbf{Periodo di \textit{slack}}

\textbf{\textit{Phablet}}

\textbf{\textit{PHP}}

\textbf{Piattaforma}

\textbf{\textit{Presenter}}

\textbf{\textit{Refactoring}}

\textbf{\textit{Responsive}}

\textbf{Robustezza}

\textbf{\textit{Run-time}}

\textbf{\textit{Server cloud}}

\textbf{\textit{Server SMTP}}

\textbf{\textit{Server web}}

\textbf{Servizio \textit{web}}

\textbf{\textit{Servlet}}

\textbf{\textit{Software} gestionale}

\textbf{Tecnologie \textit{web}}

\textbf{\textit{Template}}

\textbf{\textit{Ticketing}}

\textbf{Tracciabilità}

\textbf{\textit{Transact-SQL}}

\textbf{\textit{UML}}

\textbf{Usabilità}

\textbf{\textit{View}}