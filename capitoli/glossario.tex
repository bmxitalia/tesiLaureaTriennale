\chapter{Glossario} \label{gloss}

\textbf{\textit{Agile}} Lo sviluppo \textit{agile} è un approccio allo sviluppo del \textit{software} in base al quale i requisiti e le soluzioni si evolvono attraverso lo sforzo collaborativo di \textit{team} auto-organizzanti e interfunzionali e dei loro clienti / utenti finali. Promuove la pianificazione adattiva, lo sviluppo evolutivo, la consegna anticipata e il miglioramento continuo, inoltre incoraggia una risposta rapida e flessibile al cambiamento.\\

\textbf{\textit{AJAX}} \textit{AJAX} (\textit{Asynchronous JavaScript And XML}) è un insieme di tecniche di sviluppo \textit{web} che utilizzano tecnologie \textit{web} \textit{client-side} per creare applicazioni \textit{web} asincrone. Con \textit{AJAX}, le applicazioni \textit{web} possono inviare e recuperare dati da un \textit{server} in modo asincrono, senza interferire con la visualizzazione e il comportamento della pagina esistente.\\

\textbf{\textit{API}} In informatica, un'\textit{API} (\textit{Application Programming Interface}) è un insieme di definizioni di \textit{subroutine}, protocolli di comunicazione e strumenti per la creazione di \textit{software}. Generalmente, si tratta di un insieme di metodi di comunicazione chiaramente definiti tra vari componenti. Una buona \textit{API} semplifica lo sviluppo di un programma fornendo tutti i blocchi costitutivi, che vengono poi assemblati dal programmatore. Un'\textit{API} può essere scritta per un sistema \textit{web-based}, un sistema operativo, un sistema di \textit{database}, \textit{hardware} del \textit{computer} o una libreria \textit{software}.\\

\textbf{\textit{API testing}} L'\textit{API testing} è un tipo di test del \textit{software} che prevede il test di \textit{API} direttamente, o come parte di test di integrazione, per determinare se soddisfano le aspettative in termini di funzionalità, affidabilità, prestazioni e sicurezza.

\textbf{Applicazione \textit{web}} Un'applicazione \textit{web} (\textit{web application} in inglese) è un programma \textit{client-server} dove il \textit{client} (compresa l'interfaccia utente e la logica lato \textit{client}) viene eseguito su un \textit{browser web}.\\

\textbf{Architettura} L'architettura del \textit{software} si riferisce alle strutture di alto livello di un sistema \textit{software} e alla disciplina della loro creazione. Ogni struttura comprende elementi \textit{software}, relazioni tra loro e proprietà sia degli elementi che delle relazioni. \\

\textbf{Architettura a strati} Nell'Ingegneria del \textit{Software}, l'architettura \textit{multitier} (spesso definita architettura \textit{n-tier}) è un'architettura \textit{client-server} in cui le funzioni di presentazione, elaborazione delle applicazioni e gestione dei dati sono fisicamente separate. L'uso più diffuso dell'architettura \textit{multitier} è l'architettura a tre livelli. \\

\textbf{Attore} Nell'Ingegneria del software, il termine attore viene utilizzato all'interno dei casi d'uso per indicare il
soggetto che interagisce con il sistema. \\

\textbf{\textit{Azure}} \textit{Microsoft Azure} è un servizio di \textit{cloud computing} creato da \textit{Microsoft} per la creazione, il test, l'implementazione e la gestione di applicazioni e servizi attraverso una rete globale di \textit{data center} gestiti da \textit{Microsoft}. \\

\textbf{\textit{Back end}} Nell'Ingegneria del \textit{Software}, i termini \textit{front end} e \textit{back end} fanno riferimento alla separazione delle responsabilità tra il livello di presentazione (\textit{front end}) e il livello di accesso ai dati (\textit{back end}) di un \textit{software} o di un'infrastruttura fisica. Nel modello \textit{client-server}, il \textit{client} viene generalmente considerato come il \textit{front end} e il \textit{server} viene solitamente considerato il \textit{back end}, anche quando alcuni lavori di presentazione vengono effettivamente eseguiti sul \textit{server} stesso.\\

\textbf{\textit{Baseline}} Una \textit{baseline} è un risultato tangibile che permette di misurare il grado di avanzamento del progetto. Essa rende vera una \textit{milestone}, la quale può essere associata ad una o più \textit{baseline}. Nella gestione della configurazione, una \textit{baseline} è una base verificata, approvata e garantita, composta da un insieme di \textit{CI} (\textit{Configuration Item}) e dalla quale non si può retrocedere.\\

\textbf{\textit{Big-bang-integration}} È un test d'integrazione nel quale tutti i componenti o moduli di sistema vengono integrati e testati contemporaneamente.
In questo approccio i singoli moduli non vengono integrati fino a quando non hanno superato i corrispondenti test
d'unità, diventando pronti all'integrazione.\\

\textbf{\textit{C\#}} \textit{C\#} (pronunciato \textit{C sharp}) è un linguaggio di programmazione multi-paradigma \textit{general-purpose} che comprende una forte tipizzazione e i paradigmi: imperativo, dichiarativo, funzionale, generico, orientato agli oggetti (\textit{class-based}) e ai componenti (\textit{component-based}).\\

\textbf{Caso d'uso} Nell'Ingegneria del \textit{Software}, un caso d'uso è un elenco di azioni o eventi che tipicamente definiscono le interazioni tra un ruolo (noto in \textit{UML} come attore) e un sistema per raggiungere un obiettivo. L'attore può essere un umano o sistema esterno.\\

\textbf{Ciclo di vita} Nell'Ingegneria del \textit{Software}, un processo di sviluppo \textit{software} è il processo di divisione del lavoro di sviluppo del \textit{software} in fasi distinte, per migliorare la progettazione, la gestione del prodotto e del progetto. È anche conosciuto come ciclo di vita di sviluppo del \textit{software}.\\

\textbf{\textit{Cloud computing}} Il \textit{cloud computing} è costituito da spazi condivisi di risorse informatiche configurabili e servizi di \textit{higher-level} che possono essere rapidamente forniti con il minimo sforzo di gestione, spesso su \textit{Internet}.\\

\textbf{\textit{Code-n-fix}} Il \textit{code and fix} è una pratica molto comune nello sviluppo del \textit{software} e non è considerato un vero e proprio modello di sviluppo del \textit{software}. Esso è assimilabile ad un modello iterativo che si alterna in due fasi, il quale è quasi totalmente privo di organizzazione del processo. Infatti, il lavoro dagli sviluppatori inizia senza che essi abbiano un'idea dettagliata di cosa il programma debba fare e di come debba essere implementato. Si tratta, di conseguenza, di un modello in cui il \textit{software} si adatta progressivamente a ciò che il suo progettista desidera. \\

\textbf{Codice nativo} Il codice nativo è un codice che viene compilato per essere eseguito su un particolare processore e il suo set di istruzioni.\\

\textbf{Codice oggetto} In informativa, il codice oggetto è il prodotto di un compilatore. Generalmente, il codice oggetto è una sequenza di istruzioni in un linguaggio informatico, solitamente codice macchina.\\

\textbf{\textit{Commit}} Nei sistemi di controllo delle versioni, un \textit{commit} aggiunge le ultime modifiche al codice sorgente nel \textit{repository}.\\

\textbf{\textit{Community}} Una \textit{community} è una piccola o grande unità sociale che ha qualcosa in comune. Nel caso dello sviluppo \textit{software}, una \textit{community} è spesso un forum dove si condividono idee o si cerca di risolvere problematiche di sviluppo \textit{software} in gruppo.\\

\textbf{\textit{Compile-time}} In informatica, il tempo di compilazione si riferisce alle operazioni eseguite da un compilatore, i requisiti di un linguaggio di programmazione che devono essere soddisfatti dal codice sorgente per essere compilato correttamente, o le proprietà del programma su cui è possibile ragionare durante la compilazione.\\

\textbf{\textit{Container}} Un \textit{container} è un'unità standard di \textit{software} che racchiude il codice e tutte le sue dipendenze, in modo che l'applicazione funzioni rapidamente e in maniera affidabile su ambienti di elaborazione diversi.\\

\textbf{Correttezza per costruzione} La correttezza per costruzione è un metodo radicale, efficace ed economico per costruire software con integrità dimostrabile.\\

\textbf{\textit{Cross-compiled}} L'approccio \textit{cross-compiled} è un'approccio su cui un \textit{framework cross-platform} può basarsi per generare il codice sorgente dell'applicazione. Nello specifico, un'approccio \textit{cross-compiled} utilizza un \textit{cross-compiler}, ossia un compilatore capace di produrre codice sorgente per piattaforme diverse rispetto a quella su cui viene eseguito.\\

\textbf{\textit{CSS}} \textit{Cascading Style Sheets} (\textit{CSS}) è un linguaggio di stile usato per descrivere la presentazione di un documento scritto in un linguaggio di \textit{markup}, ad esempio \textit{HTML}. Fa parte delle tecnologie fondamentali utilizzate nel \textit{World Wide Web}.\\

\textbf{\textit{DBMS}} In informatica, un \textit{Database Management System}, abbreviato in \textit{DBMS}, è un sistema \textit{software} progettato per consentire la creazione, la manipolazione e l'interrogazione efficiente di \textit{database}, ospitato su architettura \textit{hardware} dedicata oppure su semplice \textit{computer}.\\

\textbf{\textit{Deploy}} Effettuare il \textit{deploy} di un sistema \textit{software} significa eseguire tutte le attività che rendono il sistema pronto all'utilizzo.\\

\textbf{\textit{Design pattern}} Nell'Ingegneria del Software, un \textit{design pattern} è una soluzione generale riutilizzabile per un problema che si verifica comunemente in un dato contesto della progettazione del \textit{software}.\\

\textbf{\textit{Desktop publishing}} \textit{Desktop publishing} (\textit{DTP} abbreviato) consiste nella creazione di documenti utilizzando competenze di \textit{layout} di pagina su un \textit{computer} personale principalmente per la stampa. Il \textit{software} di \textit{desktop publishing} può generare \textit{layout} e produrre testi e immagini di qualità tipografica paragonabili alla tipografia e alla stampa tradizionali.\\

\textbf{Diagramma di \textit{Gantt}} Il diagramma di \textit{Gantt} è uno strumento di supporto alla gestione dei progetti. Viene costruito partendo da un asse orizzontale - a rappresentazione dell'arco temporale totale del progetto, suddiviso in fasi incrementali (ad esempio, giorni, settimane, mesi) - e da un asse verticale - a rappresentazione delle mansioni o attività che costituiscono il progetto.\\

\textbf{\textit{Dialog}} La finestra di dialogo (\textit{dialog box}) è una piccola finestra che comunica o richiede delle informazioni all'utente.\\

\textbf{\textit{E-business}} \textit{Business online}, o \textit{e-business}, è qualsiasi tipo di transazione commerciale che include la condivisione di informazioni su \textit{Internet}.\\

\textbf{\textit{End-point}} Un \textit{end-point} di comunicazione è una tipologia di nodo di rete di comunicazione. È un'interfaccia esposta da una parte comunicante o da un canale di comunicazione.\\

\textbf{\textit{ER}} Un diagramma \textit{Entity Relationship} (ER) è un diagramma che mostra le relazioni tra le entità presenti in un \textit{database}.\\

\textbf{\textit{ERP}} \textit{Enterprise resource planning} (letteralmente ``pianificazione delle risorse d'impresa'', spesso abbreviato in \textit{ERP}) è un \textit{software} di gestione che integra tutti i processi di \textit{business} rilevanti di un'azienda (vendite, acquisti, gestione magazzino, contabilità ecc.).\\

\textbf{\textit{Framework}} Un \textit{framework} è un'architettura generica composta da un insieme di componenti \textit{software} prefabbricate che fornisce una base facilmente riusabile per una grande varietà di applicazioni in un dato dominio. Le componenti di un \textit{framework} sono bel legate tra loro e pronte all'uso.\\

\textbf{\textit{Framework cross-platform}} Un \textit{framework cross-platform} è un \textit{framework} utilizzabile per lo sviluppo \textit{cross-platform}, ossia un approccio che consente di sviluppare una singola base di codice per più piattaforme o ambienti \textit{software}.\\

\textbf{\textit{Front end}} Per la definizione di questo termine riferirsi alla definizione data per il termine \textit{back end}.\\

\textbf{Funzione anonima} In informatica, una funzione anonima è una definizione di funzione che non è associata a un identificatore.\\

\textbf{\textit{HTML}} \textit{Hypertext Markup Language} (\textit{HTML}) è il linguaggio di \textit{markup} standard per la creazione di pagine \textit{web} e applicazioni \textit{web}. Fa parte delle tecnologie fondamentali utilizzate nel \textit{World Wide Web}.\\

\textbf{\textit{HTTP}} In telecomunicazioni e informatica, l'\textit{HyperText Transfer Protocol} (\textit{HTTP}) è un protocollo a livello applicativo usato come principale sistema per la trasmissione d'informazioni sul \textit{web}, ovvero in un'architettura tipica \textit{client-server}.\\

\textbf{\textit{ICT}} La tecnologia dell'informazione e della comunicazione (\textit{ICT}) è un termine estensivo per la tecnologia dell'informazione (\textit{IT}) che sottolinea il ruolo delle comunicazioni unificate e dell'integrazione di telecomunicazioni, \textit{computer} e \textit{software} aziendale necessario, \textit{middleware}, archiviazione e sistemi audiovisivi, i quali consentono agli utenti di accedere, archiviare, trasmettere e manipolare le informazioni.\\

\textbf{\textit{IDE}} Un ambiente di sviluppo integrato (\textit{IDE}) è un'applicazione \textit{software} di supporto allo sviluppo \textit{software}. Un \textit{IDE} di solito consiste in un \textit{editor} di codice sorgente, strumenti di automazione per la \textit{build} e un \textit{debugger}.\\

\textbf{Interprete} In informatica, un interprete è un programma che esegue istruzioni scritte in un linguaggio di programmazione o di \textit{scripting}, senza richiedere che siano precedentemente compilate in linguaggio macchina.\\

\textbf{\textit{JavaServer Pages}} \textit{JavaServer Pages} (\textit{JSP}) è una tecnologia che aiuta gli sviluppatori di \textit{software} a creare pagine \textit{web} dinamiche basate su \textit{HTML}, \textit{XML} o altri tipi di documenti.\\

\textbf{\textit{JDBC}} \textit{Java Database Connectivity} (\textit{JDBC}) è un'\textit{API} per il linguaggio di programmazione \textit{Java}, che definisce in che modo un \textit{client} può accedere ad un \textit{database}.\\

\textbf{\textit{JQuery}} \textit{jQuery} è una libreria \textit{JavaScript} progettata per semplificare l'attraversamento e la manipolazione del \textit{DOM} (\textit{Document Object Model}) \textit{HTML}, nonché la gestione di eventi, animazioni e \textit{AJAX}.\\

\textbf{\textit{JSON}} In informatica, \textit{JavaScript Object Notation} (\textit{JSON}) è un formato di \textit{file} standard che utilizza un testo leggibile per trasmettere oggetti costituiti da coppie valore-attributo e tipi di dato \textit{array} (o un qualsiasi altro valore \textit{serializzabile}). Si tratta di un formato molto comune utilizzato per la comunicazione asincrona tra \textit{browser} e \textit{server}.\\

\textbf{Libreria} In informatica, una libreria è una raccolta di risorse utilizzate dai programmatori per lo sviluppo di \textit{software}. Questi possono includere dati di configurazione, documentazione, dati di aiuto, \textit{template} di messaggi, codice pre-scritto e \textit{subroutine}, classi, valori o specifiche di tipo.\\

\textbf{\textit{Media queries}} Le \textit{media queries} sono un modulo \textit{CSS3} che consente al \textit{rendering} del contenuto di adattarsi a condizioni quali la risoluzione dello schermo.\\

\textbf{\textit{Milestone}} Una \textit{milestone} è una data di calendario associata ad uno specifico insieme di \textit{baseline}, le quali rappresentano grado di avanzamento desiderabile e pianificato.\\

\textbf{\textit{Mobile first}} Il termine \textit{mobile first} è un concetto utilizzato nel \textit{web design} e nella concezione di siti \textit{web}. Consiste nel creare un sito \textit{web} a partire da una versione ottimizzata per dispositivi mobili. Pertanto, la strategia \textit{mobile first} segue la tendenza che sempre più utenti navigano in \textit{Internet} con il loro \textit{smartphone} o \textit{tablet} invece di utilizzare una postazione \textit{desktop}.\\

\textbf{\textit{Modal}} Nella progettazione di interfacce, una finestra modale è un elemento di controllo grafico subordinato alla finestra principale di un'applicazione. Crea una modalità che disabilita la finestra principale, ma la mantiene visibile sotto di essa.\\

\textbf{\textit{MoviDOC}} MoviDoc è un'applicazione sviluppata da \visione{} che permette ai venditori di accedere ai dati delle fatture emesse.\\

\textbf{Multipiattaforma} Il termine multipiattaforma, in informatica, può essere riferito ad un linguaggio di programmazione, ad un'applicazione \textit{software} o ad un dispositivo \textit{hardware} che funziona su più di un sistema operativo.\\

\textbf{\textit{MVC}} \textit{Model-View-Controller} (\textit{MVC}) è un \textit{pattern} architetturale comunemente utilizzato per lo sviluppo di interfacce utente che divide un'applicazione in tre parti interconnesse: il \textit{model}, la \textit{view} e il \textit{controller}. Il \textit{model} è la struttura dati dell'applicazione, la \textit{view} è la rappresentazione grafica dell'informazione, mentre il \textit{controller} converte gli \textit{input} ricevuti in comandi per la \textit{view} o per il \textit{model}.\\

\textbf{\textit{MVP}} \textit{Model-view-presenter} (\textit{MVP}) è una derivazione del \textit{pattern} architetturale \textit{model-view-controller} (\textit{MVC}) comunemente utilizzato per la creazione di interfacce utente. Il \textit{presenter} preleva informazioni dal \textit{model}, le processa e le visualizza sulla \textit{view}. Per la definizione di \textit{view} e \textit{model} riferirsi alla definizione del termine \textit{MVC}. \\

\textbf{\textit{Objective-C++}} \textit{Objective-C++} è un linguaggio di programmazione \textit{general-purpose} orientato agli oggetti che aggiunge la messaggistica in stile \textit{Smalltalk} al linguaggio di programmazione \textit{C}. Era il linguaggio principalmente utilizzato da \textit{Apple} per i sistemi operativi \textit{macOS} e \textit{iOS} e le rispettive \textit{API} prima dell'introduzione di \textit{Swift}.\\

\textbf{\textit{Parsing}} In informatica, il \textit{parsing} è un processo che analizza un flusso continuo di dati in ingresso (\textit{input}) in modo da determinare la sua struttura grazie ad una data grammatica formale. Un \textit{parser} è un programma che esegue questo compito.\\

\textbf{Periodo di \textit{slack}} Il tempo di \textit{slack} può essere definito come la quantità di tempo in cui un'attività può essere posticipata senza causare ritardi o incidere sulla data di completamento del progetto.\\

\textbf{\textit{Phablet}} Il \textit{phablet} è una classe di dispositivi mobili che si trova a cavallo, in termini di dimensioni, tra \textit{smartphone} e \textit{tablet}.\\

\textbf{\textit{PHP}} \textit{Hypertext Preprocessor} (\textit{PHP}) è un linguaggio di \textit{scripting} lato \textit{server} progettato per lo sviluppo \textit{web} ed utilizzato anche come linguaggio di programmazione \textit{general purpose}.\\

\textbf{Piattaforma} In informatica, il termine piattaforma indica una base \textit{software} e/o \textit{hardware} su cui sono sviluppate e/o eseguite applicazioni. Una piattaforma può essere:
\begin{itemize}
	\item \textit{hardware}: \textit{hardware} sul quale vengono fatti eseguire un certo sistema operativo e un certo insieme di programmi applicativi;
	\item \textit{operativa}: tipo di piattaforma \textit{software} che include il sistema operativo;
	\item \textit{software}: indica il tipo di \textit{framework} ed il sistema \textit{software} di base sul quale i programmi e le applicazioni sono sviluppati e/o eseguiti.
\end{itemize}

\textbf{\textit{Refactoring}} In Ingegneria del \textit{Software}, il \textit{refactoring} è una tecnica strutturata per modificare la struttura interna di porzioni di codice senza modificarne il comportamento esterno, applicata per migliorare alcune caratteristiche non funzionali del \textit{software}.\\

\textbf{\textit{Responsive}} Il \textit{design} responsivo, o \textit{responsive web design} (\textit{RWD}), indica una tecnica di \textit{web design} per la realizzazione di siti in grado di adattarsi graficamente in modo automatico al dispositivo coi quali vengono visualizzati (\textit{computer} con diverse risoluzioni, \textit{tablet}, \textit{smartphone}, cellulari, \textit{web tv}), riducendo al minimo la necessità dell'utente di ridimensionare e scorrere i contenuti.\\

\textbf{Robustezza} In termini generali, la robustezza di un \textit{software} o di un algoritmo è la sua capacità (e la capacità del programmatore) di comportarsi in modo ragionevole in situazioni impreviste, non contemplate dalle specifiche.\\

\textbf{\textit{Run-time}} In informatica, \textit{run-time} o tempo di esecuzione è il tempo durante il quale un programma è in esecuzione, in contrasto con altre fasi del ciclo di vita del programma come \textit{compile-time}, \textit{link-time} e \textit{load-time}.\\

\textbf{\textit{Server cloud}} Il \textit{server cloud} è un \textit{server} virtuale che, attraverso un software di virtualizzazione, utilizza una porzione o un sottoinsieme del \textit{server} fisico che lo ospita.\\

\textbf{\textit{Server web}} In informatica, un \textit{server web} è un'applicazione \textit{software} che, in esecuzione su un \textit{server}, è in grado di gestire le richieste di trasferimento di pagine \textit{web} di un \textit{client}, tipicamente un \textit{web browser}.\\

\textbf{Servizio \textit{web}} In informatica, un \textit{web service} (servizio \textit{web}) è un sistema \textit{software} progettato per supportare l'interoperabilità tra diversi elaboratori su di una medesima rete, ovvero in un contesto distribuito.\\

\textbf{\textit{Servlet}} Un \textit{servlet Java} è un componente \textit{software Java} che estende le funzionalità di un \textit{server}. Sebbene i \textit{servlet} possano rispondere a qualsiasi tipo di richiesta, implementano più comunemente contenitori \textit{web} per l'\textit{hosting} di applicazioni \textit{web} su \textit{server web} e pertanto si configurano come \textit{API web} di \textit{servlet} lato \textit{server}.\\

\textbf{\textit{SMTP}} \textit{Simple Mail Transfer Protocol} (\textit{SMTP}) è un protocollo per la trasmissione di \textit{e-mail}.\\

\textbf{\textit{Software} gestionale} Il \textit{software} gestionale rappresenta l'insieme dei \textit{software} che automatizzano i processi di gestione all'interno delle aziende. Essi si dividono principalmente in macro gruppi:
\begin{itemize}
	\item \textit{software} di contabilità;
	\item \textit{software} per il magazzino;
	\item \textit{software} per la produzione;
	\item \textit{software} per il \textit{budgeting};
	\item \textit{software} di gestione ed analisi finanziaria;
	\item \textit{software} dedicato.
\end{itemize}

\textbf{Tecnologie \textit{web}} Le tecnologie \textit{web} sono tecnologie utilizzate nello sviluppo \textit{web}, ossia lo sviluppo di un sito \textit{web} o un'\textit{intranet}. \\

\textbf{\textit{Ticketing}} \textit{Ticketing} è un'attività che consiste nell'assegnamento dei \textit{ticket}. Un \textit{ticket} è uno strumento di pianificazione che corrisponde ad un problema da risolvere assegnabile ad un singolo individuo.\\

\textbf{Tracciabilità} In informatica, e in particolare nell'Ingegneria del \textit{Software}, per tracciabilità (talvolta indicata con il termine inglese corrispondente \textit{traceability}) si intende la possibilità di ricostruire la relazione fra i diversi documenti prodotti nel corso di un progetto di sviluppo \textit{software}, inclusa la stessa implementazione del sistema in uno o più linguaggi di programmazione.\\

\textbf{\textit{Transact-SQL}} In informatica, \textit{Transact-SQL} (a volte abbreviato con \textit{T-SQL}) è l'estensione proprietaria del linguaggio \textit{SQL} sviluppata da \textit{Microsoft} e \textit{Sybase}.\\

\textbf{\textit{UML}} L'\textit{UML} (\textit{Unified Modeling Language}) è un linguaggio di modellazione \textit{general purpose}, di sviluppo nel campo dell'Ingegneria del \textit{Software}, inteso a fornire uno standard per visualizzare la progettazione di un sistema.\\

\textbf{Usabilità} Nell'Ingegneria del Software, l'usabilità è il grado in cui un \textit{software} può essere utilizzato da consumatori specifici per raggiungere obiettivi quantificati con efficacia, efficienza e soddisfazione in un contesto quantificato di utilizzo.