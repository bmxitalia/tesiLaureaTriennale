% !TEX encoding = UTF-8
% !TEX TS-program = pdflatex
% !TEX root = ../tesi.tex

%**************************************************************
\chapter{Conclusioni}
\label{cap:conclusioni}
%**************************************************************
In questo capitolo vengono presentate le conclusioni in merito al periodo di stage tenutosi presso l'azienda \visione{}. In particolare vengono presentati:
\begin{itemize}
	\item un consuntivo finale;
	\item il grado di soddisfacimento dei requisiti;
	\item le conoscenze acquisite;
	\item una valutazione personale sullo stage.
\end{itemize}
%**************************************************************
\section{Consuntivo finale}

Le tempistiche e le modalità di svolgimento delle attività inzialmente concordate con il tutor aziendale sono state rispettate. Lo sviluppo di moviORDER ha avuto una durata di x ore, mentre le restanti ore sono state impiegate per piccoli accorgimenti correttivi sull'applicazione, per la documentazione del codice e per la stesura di manuale utente e sviluppatore. Lo studente ha allungato di un mese il periodo di stage per l'ottenimento di una borsa di studio e durante questo periodo è stato possibile rendere moviORDER un'applicazione pronta alla vendita. Viene di seguito presentata una tabella che mostra la differenza tra i tempi di sviluppo preventivati e quelli effettivi.

{\renewcommand{\arraystretch}{2}
\begin{center}
\begin{longtable}{ | >{\arraybackslash}p{7cm} | >{\centering\arraybackslash}p{2.5cm} | >{\centering\arraybackslash}p{2.5cm} |} }
        
\hline
\textbf{Attività} & \textbf{Ore pianificate} & \textbf{Ore effettive}  \\ \hline
\endhead
Formazione assistita sui software gestionali di \visione{} e sulle applicazioni analoghe a moviORDER & 40 & 35 \\ \hline
Formazione individuale sui framework cross-platform e scelta del framework ritenuto più opportuno & 40 & 30 \\ \hline
Ridefinizione delle specifiche comprendente delle soluzioni da realizzare e delle metodologie per implementarle & 40 & 40 \\ \hline
Realizzazione dei \glossaryItem{web services} per l'interazione con il database: servizio di autenticazione, servizio di lettura dati, servizio di scrittura dati e servizio di invio e-mail & 40 & 45 \\ \hline
Realizzazione della business logic dell'applicazione & 40 & 35 \\ \hline
Realizzazione delle interfacce grafiche dell'applicazione: login, gestione carrello, aggiunta/modifica articolo e invio ordine & 40 & 35 \\ \hline
Test di scambio dati tra VisionENTERPRISE e moviORDER & 40 & 40 \\ \hline
Documentazione del codice e stesura del manuale utente e sviluppatore & 40 & 60 \\
\hline
\caption{Consuntivo finale}
\end{longtable}
\end{center}}

Come si può osservare dal consuntivo, durante lo sviluppo dell'applicazione sono state risparmiate 20 ore in totale tra le diverse attività di sviluppo. Questo perché lo stagista si è impegnato fortemente nel soddisfacimento dei requisiti funzionali obbligatori, quindi nello sviluppo del servizio, della logica applicativa e delle interfacce grafiche. Questo ha permesso il rilascio dell'applicazione prima delle ferie estive di \visione{} in modo da permettere ai clienti di testare l'applicazione in serenità. Al rientro delle ferie lo stagista ha impegnato le ore risparmiate per la correzione di tutti i bug riscontrati dai clienti, per la documentazione del codice e per la stesura dei manuali dell'applicazione.

%**************************************************************
\section{Grado di soddisfacimento dei requisiti}

%**************************************************************
\section{Conoscenze acquisite}

Il periodo di stage ha permesso allo stagista di apprendere diverse nuove tecnologie e di approfondire l'utilizzo di tecnologie già in parte conosciute. Vengono di seguito presentate le tecnologie imparate durante lo stage.

\subsection{JavaScript}

Uno dei criteri per i quali lo stagista aveva scelto \visione{} era l'utilizzo di JavaScript per lo sviluppo dell'applicazione. Questo perché il linguaggio JavaScript è sempre più richiesto dalle aziende e lo stagista ne aveva solamente una conoscenza basilare. Durante il periodo di formazione è stato possibile imparare ogni dettaglio di JavaScript puro seguendo ogni tutorial presente sulla guida del sito web W3C. Questo ha permesso di prendere confidenza con il linguaggio e di realizzare la logica applicativa di moviORDER velocemente. In particolare, lo studente ha appreso l'utilizzo di AJAX come efficace strategia per inviare richieste HTTP tramite JavaScript.

\subsection{SQL Server}

Prima dello stage lo stagista conosceva solamente il DBMS \textit{mySQL} perché imparato durante il corso di Tecnologie Web. Il periodo di stage ha permesso la conoscenza di SQL Server, ampliando la conoscenza dello studente in termini di gestione di database.

\subsection{Apache Tomcat e oggetti servlet}

Prima dello stage lo studente non aveva mai realizzato un servizio web e non aveva idea di cosa fossero gli oggetti servlet e di quali fossero le loro applicazioni. Lo stage ha permesso la comprensione delle tecniche per:
\begin{itemize}
	\item la realizzazione di un servizio web scritto in Java tramite l'utilizzo di oggetti servlet;
	\item il deploy di un servizio sul server web Apache Tomcat.
\end{itemize}

\subsection{PhoneGap}

Prima dello stage lo studente non aveva mai realizzato un'applicazione mobile. Lo stage ha permesso allo studente di apprendere il framework cross-platform PhoneGap e di utilizzarlo per la realizzazione di applicazioni mobile su più piattaforme, in particolare iOS e Android. Si è appreso inoltre l'utilizzo delle funzionalità di base dei software Android Studio e XCode.

%**************************************************************
\section{Valutazione personale}
