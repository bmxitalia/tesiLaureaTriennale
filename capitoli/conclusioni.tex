% !TEX encoding = UTF-8
% !TEX TS-program = pdflatex
% !TEX root = ../tesi.tex

%**************************************************************
\chapter{Conclusioni}
\label{cap:conclusioni}
%**************************************************************
In questo capitolo vengono presentate le conclusioni tratte dallo stagista in merito al periodo di \textit{stage} tenutosi presso l'azienda \visione{}. In particolare vengono presentati:
\begin{itemize}
	\item un consuntivo finale;
	\item il grado di soddisfacimento dei requisiti al termine del progetto;
	\item le conoscenze acquisite;
	\item gli sviluppi futuri proposti;
	\item una valutazione personale sullo \textit{stage}.
\end{itemize}
%**************************************************************
\section{Consuntivo finale}

Le tempistiche e le modalità di svolgimento delle attività inizialmente concordate con il \textit{tutor} aziendale sono state rispettate. Lo sviluppo di \textit{moviORDER} ha avuto una durata di 260 ore e le restanti 60 ore sono state impiegate per accorgimenti correttivi sull'applicazione, per la documentazione del codice e per la stesura del manuale utente e sviluppatore. Per ottenere una borsa di studio lo studente ha allungato di un mese il periodo di \textit{stage}, durante il quale è stato possibile migliorare il \textit{front end} di \textit{moviORDER}. Viene di seguito presentata una tabella che mostra la differenza tra i tempi di sviluppo preventivati e quelli effettivi.
\newpage
{\renewcommand{\arraystretch}{2}
\begin{center}
\begin{longtable}{ | >{\arraybackslash}p{7cm} | >{\centering\arraybackslash}p{2cm} | >{\centering\arraybackslash}p{2cm} |} 
\hline
\textbf{Attività} & \textbf{Ore pianificate} & \textbf{Ore effettive}  \\ \hline
\endhead
Formazione assistita sui \textit{software} gestionali di \visione{} e sulle applicazioni analoghe a \textit{moviORDER} & 40 & 35 \\ \hline
Formazione individuale sui \textit{framework cross-platform} e scelta del \textit{framework} ritenuto più opportuno & 40 & 30 \\ \hline
Ridefinizione delle specifiche comprendente delle soluzioni da realizzare e delle metodologie per implementarle & 40 & 40 \\ \hline
Realizzazione dei \textit{web services} per l'interazione con il \textit{database}: servizio di autenticazione, servizio di lettura dati, servizio di scrittura dati e servizio di invio e-mail & 40 & 45 \\ \hline
Realizzazione della \textit{business logic} dell'applicazione & 40 & 35 \\ \hline
Realizzazione delle interfacce grafiche dell'applicazione: login, gestione carrello, aggiunta/modifica articolo e invio ordine & 40 & 35 \\ \hline
Test di scambio dati tra \textit{VisionENTERPRISE} e \textit{moviORDER} & 40 & 40 \\ \hline
Documentazione del codice e stesura del manuale utente e sviluppatore & 40 & 60 \\
\hline
\caption{Consuntivo finale}
\end{longtable}
\end{center}}

Come si può osservare dal consuntivo, durante lo sviluppo dell'applicazione sono state risparmiate 20 ore tra le diverse attività. Questo perché lo stagista si è impegnato fortemente nel soddisfacimento dei requisiti funzionali obbligatori e quindi nello sviluppo del servizio, della logica applicativa e delle interfacce grafiche. Ciò ha permesso il rilascio dell'applicazione prima delle ferie estive di \visione{}, in modo da consentire al \textit{tutor} aziendale di testare l'applicazione in serenità. Al rientro delle ferie lo stagista ha impegnato le ore risparmiate per la correzione dei \textit{bug} riscontrati, per la documentazione del codice e per la stesura dei manuali dell'applicazione.

%**************************************************************
\section{Grado di soddisfacimento dei requisiti}

Al termine dello \textit{stage} sono stati soddisfatti tutti i requisiti richiesti tranne il requisito (RVF9), poiché ritenuto dall'azienda di importanza irrilevante. Tale requisito richiedeva che l'applicazione permettesse la firma digitale del cliente a fine ordine. Viene di seguito fornita la tabella che illustra il grado di soddisfacimento dei requisiti.

{\renewcommand{\arraystretch}{2}
\begin{center}
\begin{longtable}{ | >{\arraybackslash}p{4cm} | >{\centering\arraybackslash}p{4cm} | >{\centering\arraybackslash}p{4cm} | }
\hline
\textbf{Tipologia} & \textbf{Numero requisiti} & \textbf{Soddisfatti} \\ \hline
\endhead
\textbf{Funzionale} & 102 & 102 \\ \hline
\textbf{Qualitativo} & 2 & 2 \\ \hline
\textbf{Di vincolo} & 9 & 8 \\ \hline
\textbf{Totale} & 113 & 112 \\ \hline
\caption{Grado di soddisfacimento dei requisiti}
\end{longtable}
\end{center}}

%**************************************************************
\section{Conoscenze acquisite}

Il periodo di \textit{stage} ha permesso l'apprendimento di nuove tecnologie e l'approfondimento di alcune già in parte conosciute. Vengono di seguito presentate le tecnologie apprese durante lo \textit{stage}.

\subsection{JavaScript}

Uno dei criteri per i quali lo stagista aveva scelto \visione{} era l'utilizzo di \textit{JavaScript} nello sviluppo del progetto. Questo perché il linguaggio \textit{JavaScript} è sempre più richiesto dalle aziende e lo stagista ne aveva solamente una conoscenza basilare. Durante il periodo di formazione è stato possibile imparare ogni dettaglio di \textit{JavaScript} puro, seguendo i \textit{tutorial} presenti sulla guida del sito \textit{web} \textit{W3C}. Questo ha permesso di prendere confidenza con il linguaggio e di realizzare la logica applicativa velocemente. In particolare lo studente ha appreso l'utilizzo di \textit{AJAX} come efficace strategia per inviare richieste \textit{HTTP} tramite \textit{JavaScript}.

\subsection{SQL Server}

Prima dello \textit{stage} lo stagista conosceva solamente il \textit{DBMS} \textit{mySQL} perché imparato durante il corso di Tecnologie \textit{Web}. Il periodo di \textit{stage} ha permesso la comprensione di \textit{SQL Server}, ampliando la conoscenza dello studente in termini di gestione di \textit{database}.

\subsection{Apache Tomcat e oggetti servlet}

Prima dello \textit{stage} lo studente non aveva mai realizzato un servizio \textit{web} e non aveva idea di cosa fossero gli oggetti \textit{servlet}, di quali fossero le loro applicazioni e di come rendere un servizio \textit{web} operativo. Lo \textit{stage} ha permesso la comprensione delle tecniche per:
\begin{itemize}
	\item la realizzazione di un servizio \textit{web} scritto in \textit{Java} tramite l'utilizzo di oggetti \textit{servlet};
	\item il \glossaryItem{deploy} di un servizio sul \textit{server web} \textit{Apache Tomcat}.
\end{itemize}

\subsection{PhoneGap}

Prima dello \textit{stage} lo studente non aveva mai realizzato un'applicazione \textit{mobile}. Lo \textit{stage} ha permesso di apprendere il \textit{framework cross-platform PhoneGap} e di utilizzarlo per la realizzazione di applicazioni \textit{mobile} su più piattaforme, in particolare \textit{iOS} e \textit{Android}. Si è appreso inoltre l'utilizzo delle funzionalità di base dei \textit{software} \textit{Android Studio} e \textit{XCode}.

\section{Sviluppi futuri}

Lo stagista ritiene che l'interfaccia grafica realizzata sia attualmente utilizzabile esclusivamente come prototipo per la preparazione di demo da mostrare a dei potenziali clienti. Si ritiene necessario migliorare il \textit{design} dell'interfaccia sia in termini di usabilità che in termini di resa grafica. 
Si osserva inoltre che la scelta di un \textit{framework cross-platform} per lo sviluppo di un'applicazione che utilizza i sensori del dispositivo non è del tutto corretta. Astrarre il codice nativo per la gestione della fotocamera tramite un'interfaccia \textit{JavaScript} diminuisce le performance dell'applicazione, sia in termini di memoria che in termini di consumo energetico. In questo caso si sarebbe dovuto scrivere tale codice in linguaggio nativo. Si consiglia quindi di effettuare il \glossaryItem{refactoring} di tale parte dell'applicazione in codice nativo.
Infine, avendo utilizzato il \textit{framework PhoneGap}, si sono dovute utilizzare tecnologie \textit{web} per lo sviluppo dell'applicazione. Si può quindi osservare che tramite piccoli accorgimenti sul codice del progetto è possibile realizzare una versione \textit{web} di \textit{moviORDER}, andando a completare l'intera piattaforma.

%**************************************************************
\section{Valutazione personale}

Lo stagista ritiene che l'esperienza di \textit{stage} sia un'attività essenziale per comprendere il significato del lavoro. Non avendo mai lavorato era difficile capire le differenze tra il mondo universitario e quello lavorativo. Lo \textit{stage} ha permesso di cogliere queste differenze e di cimentarsi in un'esperienza completamente nuova. Inoltre, per un laureato in informatica è importante avere esperienza lavorativa nel proprio bagaglio personale e lo \textit{stage} obbligatorio ha permesso l'introduzione di questa.

Il periodo di \textit{stage} presso \visione{} ha permesso allo studente di calarsi nel mondo lavorativo e di cimentarsi nella realizzazione di un vero progetto \textit{software}, il quale prodotto verrà utilizzato quotidianamente dai clienti dell'azienda. Questo ha permesso di lavorare all'interno con un \textit{team} di sviluppo e di comprendere il significato di responsabilità. Il progetto del corso di Ingegneria del \textit{Software} aveva solamente fornito le basi di cosa significhi lavorare in modo responsabile. Grazie allo \textit{stage} è stato possibile avere dei veri e propri \textit{task}, assegnati da un titolare e da completare entro una certa scadenza.

La preparazione del \textit{team} di \visione{} e la serenità dell'ambiente di lavoro hanno permesso fin da subito di abbattere ansie e insicurezze antecedenti al periodo di \textit{stage}, permettendo di iniziare a lavorare in modo disciplinato e responsabile fin dall'inizio. Questo è stato possibile anche perché l'offerta di \textit{stage} era d'interesse per lo studente e ciò ha permesso di non vedere il lavoro con un ottica di impegno universitario ma con un ottica di opportunità per crescere tecnicamente ed emotivamente.

Le relazioni instaurate con il \textit{team} di sviluppo e il \textit{tutor} aziendale, gli impegni presi, il rispetto delle scadenze e degli orari lavorativi hanno incrementato le capacità organizzative e collaborative dello studente, che saranno sicuramente utili nelle prossime esperienze lavorative.