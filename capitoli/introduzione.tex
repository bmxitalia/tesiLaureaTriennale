% !TEX encoding = UTF-8
% !TEX TS-program = pdflatex
% !TEX root = ../tesi.tex

%**************************************************************
\chapter{Introduzione}
\label{cap:introduzione}
%**************************************************************

Al giorno d'oggi la tecnologia sta ricoprendo un ruolo importante nei task di tutti i giorni. In particolare, è indispensabile che un'azienda venditrice di un qualsiasi tipo di prodotti abbia una propria piattaforma online per la gestione degli ordini. Questo perché le persone si stanno abituando sempre di più ad effetturare ordini online. Infatti acquistare online risulta vantaggioso per molteplici motivi, quali il risparmio di tempo e di denaro e soprattutto la comodità di non doversi muovere da casa per acquistare un prodotto. Per molte aziende non cogliere questo cambiamento potrebbe essere fallimentare, infatti, in futuro, la maggior parte degli acquisti avverrà principalmente online per ogni tipologia di prodotto.

MoviORDER nasce per rispondere a questa necessità, proponendosi come piattaforma universale per la registrazione e l'invio di ordini online. Una qualsiasi azienda interessata a vendere i propri prodotti online può contattare \visione{} per ricevere moviORDER. Dopo un breve periodo di scambio di informazioni riguardati clienti e prodotti, moviORDER sarà pronta a ricevere ordini online dagli utenti registrati. 
I vantaggi di moviORDER sono:
\begin{itemize}
	\item per i clienti:
	\begin{itemize}
		\item possibilità di trovare i prodotti online e non solo nel negozio fisico;
		\item possibilità di risparmiare tempo e denaro dovuto al raggiungimento del negozio fisico;
		\item possibilità di avere controllo della disponibilità dei prodotti in tempo reale.
	\end{itemize}
	\item per le aziende:
	\begin{itemize}
		\item riduce il rischio di fallimento dovuto al cambiamento delle convenzioni degli utenti;
		\item aumenta il numero di clienti per l'azienda, permettendo a coloro che non riescono a raggiungere il negozio fisico, di poter comunque effettuare ordini grazie alla piattaforma online.
	\end{itemize}
\end{itemize}

%**************************************************************
\section{L'azienda}

\visione{} è un'azienda nuova che da 30 anni si occupa di informatica e più precisamente di quella parte dell'informatica dedicata alle applicazioni gestionali. Inizialmente l'attività di \visione{} era dedicata ad aziende, enti pubblici, studi professionali e centri di elaborazione dati, gestendo totalmente problematiche informatiche, progettazione di sistemi, hardware, reti, sistemi operativi e software applicativo. Oggi \visione{} punta sulla specializzazione, dedicandosi in modo particolare allo sviluppo del software applicativo e dei relativi servizi di implementazione dello stesso nell'azienda. \visione{} è un team di persone esperte e motivate che opera direttamente su gran parte del Nord Est e indirittamente sull’intero territorio nazionale. \visione{} si rivolge a piccole e medie aziende italiane che intendono impostare sul sistema informatico non solo la semplice gestione amministrativa o di magazzino, ma la completa organizzazione aziendale per affrontare un futuro sempre più complesso e veloce con il supporto di un sistema informatico che aiuti l’azienda a prendere decisioni sempre basate su dati precisi.

\begin{figure}[!h] 
    \centering 
    \includegraphics[width=0.6\columnwidth]{visione/logoVisione} 
    \caption{Logo dell'azienda \visione{}}
\end{figure}

\subsection{Core business}

\visione{} ha principalmente due core business: VisionENTERPRISE e movidat. VisionENTERPRISE è un software gestionale \glossaryItem{ERP} dedicato alle medie e piccole aziende industriali, commerciali e dei servizi, che gestiscono notevoli moli di dati e hanno la necessità di lavorare con grande velocità e stabilità. Il software è in grado di collegarsi a tutte le informazioni dell'azienda cliente e di interfacciarsi con tutti i software utilizzati al fine di gestire in modo ottimale l'intera organizzazione aziendale con la massima semplicità e velocità operativa. Il software è in grado di rendere disponibile in tempo reale, alla direzione o al titolare dell'azienda, tutte le informazioni di cui hanno bisogno per prendere decisioni sulla base di dati concreti e oggettivi. Altra qualità di VisionENTERPRISE è la sua completa copertura funzionale: dalla contabilità al magazzino, dall'area commerciale alla produzione, dal controllo di gestione all'E-business.

\begin{figure}[!h] 
    \centering 
    \includegraphics[width=0.8\columnwidth]{visione/visionEnterprise} 
    \caption{Banner del software VisionENTERPRISE}
\end{figure}

Movidat è un marchio di \visione{} che progetta e sviluppa applicazioni per dispositivi mobile rivolte alle piccole e medie imprese che vogliono rendere i loro processi più semplici, veloci ed efficienti. Lo slogan di movidat è ``ovunque tu sia, il tuo business a portata di mano!'', infatti le applicazioni movidat sono rivolte ai dipendenti delle aziende che lavorano in movimento e che devono avere sempre tutto sotto controllo ed essere in grado di agire tempestivamente in caso di problematiche inaspettate. Le soluzioni movidat sono compatibili con la maggior parte dei software gestionali disponibili sul mercato. Tra le soluzioni di meglior successo vi sono:
\begin{itemize}
	\item \textbf{moviCheck}: applicazione rivolta alle aziende che vogliono offrire alle proprie figure direzionali uno strumento in grado di analizzare in mobilità i dati di business più significativi;
	\item \textbf{moviSell}: applicazione specificatamente rivolta ai professionisti della vendita, ideata come supporto per la gestione dei rapporti con i clienti.
\end{itemize}

\begin{figure}[!h] 
    \centering 
    \includegraphics[width=0.6\columnwidth]{visione/movidat} 
    \caption{Logo del marchio movidat}
\end{figure}

MoviORDER rientra tra le applicazioni del marchio movidat andando a soddisfare i bisogni del cliente finale dell'azienda. Infatti, moviORDER permette ai clienti di un'azienda di acquistare i prodotti della stessa direttamente dall'applicazione, installata sugli smartphone degli stessi.

%**************************************************************
\section{Offerta di stage}

Il 10 Aprile 2018 si è tenuta a Padova la 15esima edizione di Stage-IT, iniziativa che mira ad agevolare l'incontro tra aziende e studenti universitari che puntano ad entrare nel mondo del lavoro con specifico riferimento al settore ICT, favorendo un'occasione di conoscenza reciproca tramite colloqui individuali. Le aziende partecipanti propongono spesso progetti innovativi con ottime opportunità di accrescimento delle competenze individuali. Proprio per questo motivo, lo stagista ha deciso di partecipare attivamente all'evento concludendo positivamente sette colloqui in totale. 

\begin{figure}[!h] 
    \centering 
    \includegraphics[width=0.8\columnwidth]{loghi/stage} 
    \caption{Logo dell'evento Stage-IT}
\end{figure}

Le aziende sono state scelte dallo studente per ambito e tecnologie di sviluppo adottate nel progetto. In particolare, lo studente era in cerca di:
\begin{itemize}
	\item progetti di sviluppo in ambito mobile (preferibilmente Android): al giorno d'oggi i task si stanno spostando sempre di più dalle postazioni desktop ai dispositivi portatili;
	\item progetti di sviluppo di applicazioni web (preferibilmente con utilizzo di framework e librerie Javascript moderne): le skill di programmazione in Javascript sono richieste da gran parte delle aziende che si occupano della realizzazione di applicazioni web.
\end{itemize}
Tra le varie offerte di stage, \visione{} rientrava in entrambe le preferenze, infatti proponeva un progetto di sviluppo di un'applicazione mobile tramite l'utilizzo di un framework cross-platfom. Nonostante lo stage non prevedesse lo sviluppo in codice nativo Android, permetteva comunque lo sviluppo di un'applicazione mobile. Inoltre, richiedendo l'utilizzo di un framework corss-platform per lo sviluppo era pressoché implicito l'utilizzo di tecnologie web, tra le quali anche Javascript. Per cui, il buon compromesso tra tecnologie conosciute e tecnologie ritenute interessanti ha favorito \visione{} tra le varie offerte analizzate.

\section{Obiettivi e pianificazione}

\section{Rischi}

%**************************************************************
\section{Organizzazione del testo}

\begin{description}
    \item[{\hyperref[cap:processi-metodologie]{Il secondo capitolo}}] descrive ...
    
    \item[{\hyperref[cap:descrizione-stage]{Il terzo capitolo}}] approfondisce ...
    
    \item[{\hyperref[cap:analisi-requisiti]{Il quarto capitolo}}] approfondisce ...
    
    \item[{\hyperref[cap:progettazione-codifica]{Il quinto capitolo}}] approfondisce ...
    
    \item[{\hyperref[cap:verifica-validazione]{Il sesto capitolo}}] approfondisce ...
    
    \item[{\hyperref[cap:conclusioni]{Nel settimo capitolo}}] descrive ...
\end{description}

Riguardo la stesura del testo, relativamente al documento sono state adottate le seguenti convenzioni tipografiche:
\begin{itemize}
	\item gli acronimi, le abbreviazioni e i termini ambigui o di uso non comune menzionati vengono definiti nel glossario, situato alla fine del presente documento;
	\item per la prima occorrenza dei termini riportati nel glossario viene utilizzata la seguente nomenclatura: \emph{parola}\glsfirstoccur;
	\item i termini in lingua straniera o facenti parti del gergo tecnico sono evidenziati con il carattere \emph{corsivo}.
\end{itemize}